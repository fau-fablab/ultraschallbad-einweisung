%%%%%%%%%%%%%%%%%%%%%%%%%%%%%%%%%%%%%%%%%%%%%%%%
% COPYRIGHT: (C) 2012-2015 FAU FabLab and others
% Bearbeitungen ab 2015-02-20 fallen unter CC-BY-SA 3.0
% Sobald alle Mitautoren zugestimmt haben, steht die komplette Datei unter CC-BY-SA 3.0. Bis dahin ist der Lizenzstatus aller alten Bestandteile ungeklärt.
%%%%%%%%%%%%%%%%%%%%%%%%%%%%%%%%%%%%%%%%%%%%%%%%


\newcommand{\basedir}{fablab-document}
\documentclass{\basedir/fablab-document}

% \usepackage{fancybox} %ovale Boxen für Knöpfe - nicht mehr benötigt
\usepackage{amssymb} % Symbole für Knöpfe
\usepackage{subfigure,caption}

\usepackage{marvosym} % für Briefumschlag-Symbol
\usepackage{eurosym}
\usepackage{tabularx} % Tabellen mit bestimmtem Breitenverhältnis der Spalten
\usepackage{multirow} % Tabellen Zellen die sich über mehrere Zeilen ausdehnen
\usepackage{wrapfig} % Textumlauf um Bilder
\renewcommand{\texteuro}{\euro}

\linespread{1.2}

%\author{FAU FabLab}
\title{Einweisung Ultraschallbad}

\newcommand{\todo}[1]{\textbf{\color{red}{TODO: #1}}}


\begin{document}
	\maketitle
	
\begin{section}{Gefahren}

Für dich:

 Knochen und so

Dämpfe von Lösungsmitteln o.ä.

unbeabsichtigtes Einschalten


Für das Gerät:

Gerät hält keine Säure/Lauge aus

Zu geringe Befüllung zerstört die Ultraschallwandler. Ebenso kann das Gerät kaputtgehen, wenn andere Flüssigkeiten verwendet werden oder Teile ohne den Einhängekorb direkt auf dem Boden liegen.

\end{section}


\begin{section}{Regeln}
Personen ohne unterschriebene Einweisung dürfen dieses Gerät nicht benutzen, auch nicht unter Anleitung. Zuerst ist die Einweisung zu unterschreiben.

Gerät bis zur Markierung mit Wasser füllen. Teile in Einhängekorb legen. diesen so in das Gerät einsetzen, dass er nicht auf dem Boden aufsetzt.

Nur Wasser und Ultraschallreiniger verwenden, ersatzweise statt Ultraschallreiniger auch Universalreiniger-Konzentrat oder Spülmittel. Andere Stoffe sind nicht erlaubt, weder Lösemittel, Alkohol, noch Säuren, Laugen oder sonstige Mittel.

Für jegliche Arbeiten (Befüllen, Teile einlegen, und so weiter) den \textbf{roten} Hauptschalter ausschalten. Der Timer-Knopf ist nicht ausreichend. Auf keinen Fall hineinlangen, während der Hauptschalter an ist!

Im Betrieb regelmäßig (etwa alle 10 Minuten) nach dem Bad schauen und Füllstand kontrollieren. Beheizt nur mit Timer verwenden, nicht auf dauer-an. Deckel aufsetzen, damit fremde Personen nicht hineinlangen.

Empfohlener Aufstellungsort ist rechts vom Waschbecken, da das Gerät keine Spritzer des Ätzbades aushält.

Nach Abschluss Kabel ausstecken. Wasser nach links hinten ausleeren, sodass es nicht auf die Bedienknöpfe oder den Stromanschluss kommt. Gerät ausspülen, trockenwischen und in die Schublade zurückräumen.

Der Ultraschallreiniger kostet Geld, siehe Etikett. Wir bitten außerdem um eine Spende für die Instandhaltung des Ultraschallbads, es war wirklich nicht günstig.
\end{section}


\begin{section}{Tips}
Anfangs den Ultraschall im Pulsbetrieb (50 \%) verwenden und ab und zu kurz ausschalten, damit sich Luftblasen lösen.
\end{section}

 \begin{section}{Quellen}
  \begin{itemize}
   \item Anleitung Ultraschallbad TODO Name
   \item \href{http://bgi850-0.vur.jedermann.de/index.jsp?isbn=bgi850-0&alias=bgc_bi850_0_bi850_0_s5_2_21_}{BGI 850 5.2.21}
   \item \href{http://www.arbeitssicherheit.de/de/html/library/document/4989034,37}{TODO irgendwas}
  \end{itemize}

 \end{section}

\ccLicense{ultraschallbad-einweisung}{Einweisung Ultraschallbad}

\end{document}
